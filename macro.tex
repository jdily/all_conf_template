%!TEX root = algo.tex
%%% Packages use.
% \usepackage[labelfont=bf,textfont=it]{caption}
\usepackage{enumitem}
% \usepackage{algorithm2e}
\usepackage{amsmath}
% \usepackage{amssymb}
% \usepackage{amsthm}
\usepackage{amsfonts}
\usepackage{multirow}
% \usepackage{subfigure}
\usepackage{color}
\usepackage{booktabs}
\usepackage{ifthen}
\usepackage{hyperref}
% \usepackage{subfig}
\usepackage{subcaption}
\usepackage{bbold}
% \usepackage{algorithm}
% \usepackage{algorithmicx}
% \usepackage{algpseudocode}
% \usepackage[linesnumbered,ruled,vlined]{algorithm2e}
% \usepackage{coloremoji}
% \usepackage[demo]{graphicx}
% \usepackage{subfig}

% \usepackage{subfig}
% \usepackage{svg}

\usepackage{xcolor}% http://ctan.org/pkg/xcolor
% \usepackage{todonotes}
\usepackage[normalem]{ulem} % for sout


% \usepackage{algorithm}
% \usepackage{algorithmicx}
% \usepackage{algpseudocode}
% \algnewcommand\algorithmicinput{\textbf{Input:}}
% \algnewcommand\INPUT{\item[\algorithmicinput]}
% \algnewcommand\algorithmicoutput{\textbf{Output:}}
% \algnewcommand\OUTPUT{\item[\algorithmicoutput]}
% \algnewcommand\algorithmicforeach{\textbf{for each}}


% \algdef{S}[FOR]{ForEach}[1]{\algorithmicforeach\ #1\ \algorithmicdo}
% \algrenewcommand{\alglinenumber}[1]{\color{red!80!blue}\footnotesize#1:}
% \renewcommand{\algorithmiccomment}[1]{\hfill$\triangleright$\textcolor{blue}{#1}}
% \algnewcommand\Func[2]{\textcolor{green}{#1}\textcolor{green}{(#2)}}
% \algnewcommand\Insert[2]{Insert {#1} to #2.}
% \algnewcommand\Input[1]{\State \textbf{Input: } #1}
% \algnewcommand\Output[1]{\State \textbf{Output: } #1}
% \renewcommand{\algorithmicrequire}{\textbf{Input:}}
% \renewcommand{\algorithmicensure}{\textbf{Output:}}
% \algnewcommand\algorithmicforeach{\textbf{for each:}}
% \algnewcommand\ForEach{\item[ \algorithmicforeach]}

% \renewcommand{\algorithmiccomment}[1]{%
%   \hfill\#\ \eqparbox{COMMENT}{#1}%
% }
% \usepackage{etoolbox}  % patch def of algorithmic environment
% \makeatletter
% \patchcmd{\algorithmic}{\addtolength{\ALC@tlm}{\leftmargin} }{\addtolength{\ALC@tlm}{\leftmargin}}{}{}
% \makeatother

% \usepackage{subfig} % for subfloat

%%% Color definition.
\definecolor{gray}{rgb}{0.5,0.5,0.5}
\definecolor{green}{rgb}{0, 0.6, 0}
\definecolor{orange}{rgb}{1, 0.5, 0}
\definecolor{mahogany}{rgb}{0.75, 0.25, 0.0}
\definecolor{purple}{rgb}{0.6, 0, 0.6}
\definecolor{darkgreen}{rgb}{0, 0.3, 0}
\definecolor{orange}{rgb}{1, 0.5, 0.}

% \newcommand{\ichao}[1]{\textcolor{blue}{\textbf{ichao: #1}}}
\newcommand{\ichao}[1]{\textcolor{blue}{{#1}}}
\newcommand{\iccmt}[1]{\textcolor{purple}{\textbf{ichao:} #1}}
\newcommand{\toby}[1]{\textcolor{green}{{#1}}}
\newcommand{\tccmt}[1]{\textcolor{green}{\textbf{toby:} #1}}
\newcommand{\todo}[1]{\textcolor{red}{\textbf{TODO:} #1}\\}
\newcommand{\discuss}[1]{\textcolor{orange}{\textbf{#1}}}
\newcommand{\upd}[1]{\colorbox{yellow}{#1}}
\newcommand{\expect}{\mathop{\mathbb{E}}\nolimits}
\newcommand{\sj}[1]{\textcolor{green}{\textbf{sj: #1}}}
%%% Editing comments.
% ignore this
\newcommand{\ignore}[1]{}
\newcommand{\none}[1]{}
\newcommand{\com}[1]{}
% comment
\newcommand{\cmt}[1]{\begin{sloppypar}\large\textcolor{red}{#1}\end{sloppypar}}
% the note in the paper
% \newcommand{\note}[1]{\cmt{Note: #1}}
% todo list
\newcommand{\note}[1]{\textcolor{red}{#1}}
\newcommand{\torevise}[1]{\textcolor{mahogany}{#1}}
%%% Editing comment.
% josh
% \newcommand{\ichao}[1]{\textcolor{blue}{#1}}

%%% Frequently used terms.
\newcommand{\etal}{{\it{et~al.}}}
\newcommand{\ie}{i.e.}
\newcommand{\eg}{e.g.}
\newcommand{\figname}{Figure}
\newcommand{\tabname}{Table}
\newcommand{\secname}{Section}
\newcommand{\algoname}{Algorithm}
\newcommand{\eqname}{Eq.}
\newcommand{\chapname}{Chapter}
\DeclareMathOperator*{\argmin}{arg\,min}
