%!TEX root = algo.tex
%%% Packages use.
% \usepackage[labelfont=bf,textfont=it]{caption}
\usepackage{enumitem}
\usepackage{amsmath}
\usepackage{amsfonts}
\usepackage{multirow}
\usepackage{bigdelim}
% \usepackage{subfigure}
\usepackage{color}
\usepackage{booktabs}
\usepackage{ifthen}
% \usepackage{hyperref} % comment out for AAAI
% \usepackage{subfig}
\usepackage{subcaption}
\usepackage{bbold}
% \usepackage[noabbrev]{cleveref}
\ifthenelse{\equal{\conf}{cvpr}}
{
}
{
\usepackage[noabbrev,capitalise,nameinlink]{cleveref}
\creflabelformat{equation}{#2\textup{#1}#3}% 
}

% \usepackage{todonotes}
\usepackage{xcolor, pgf}
\usepackage{color, colortbl}
% \usepackage[normalem]{ulem} % for sout  % comment out for AAAI
\usepackage{placeins}
\usepackage{soul}
\usepackage[skins]{tcolorbox}


\usepackage{algorithm}
\usepackage{algorithmicx}
\usepackage{algpseudocode}
\algnewcommand\algorithmicinput{\textbf{Input:}}
\algnewcommand\INPUT{\item[\algorithmicinput]}
\algnewcommand\algorithmicoutput{\textbf{Output:}}
\algnewcommand\OUTPUT{\item[\algorithmicoutput]}
\algnewcommand\algorithmicforeach{\textbf{for each}}
\algtext*{EndFor}% Remove "end for" text

\algdef{S}[FOR]{ForEach}[1]{\algorithmicforeach\ #1\ \algorithmicdo}
\algrenewcommand{\alglinenumber}[1]{\color{red!80!blue}\footnotesize#1:}
\renewcommand{\algorithmiccomment}[1]{\hfill$\triangleright$\textcolor{blue}{#1}}
\algnewcommand\Func[2]{\textcolor{green}{#1}\textcolor{green}{(#2)}}
\algnewcommand\Insert[2]{Insert {#1} to #2.}
\algnewcommand\Input[1]{\State \textbf{Input: } #1}
\algnewcommand\Output[1]{\State \textbf{Output: } #1}
\renewcommand{\algorithmicrequire}{\textbf{Input:}}
\renewcommand{\algorithmicensure}{\textbf{Output:}}

\newlength\myboxwidth
\setlength{\myboxwidth}{\dimexpr\textwidth-2\fboxsep}

%%% Color definition.
\definecolor{gray}{rgb}{0.5,0.5,0.5}
\definecolor{green}{rgb}{0, 0.6, 0}
\definecolor{orange}{rgb}{1, 0.5, 0}
\definecolor{mahogany}{rgb}{0.75, 0.25, 0.0}
\definecolor{purple}{rgb}{0.6, 0, 0.6}
\definecolor{darkgreen}{rgb}{0, 0.3, 0}
\definecolor{orange}{rgb}{1, 0.5, 0.}
\definecolor{lightblue}{rgb}{0.52, 0.75,0.91}
\definecolor{softgreen}{rgb}{0.66,0.87,0.74}
\definecolor{softred}{rgb}{0.96,0.71,0.69}
\newcommand{\ichao}[1]{\textcolor{blue}{{#1}}}

%% some useful command for discussion and leaving notes on the paper.
\DeclareRobustCommand{\xxcmt}[1]{
  \begingroup
  \definecolor{hlcolor}{RGB}{168,221,188}\sethlcolor{hlcolor}%
  \hl{\textbf{xx:} #1}%
  \endgroup
}
\DeclareRobustCommand{\todo}[1]{
  \begingroup
  \definecolor{hlcolor}{RGB}{245,183,177}\sethlcolor{hlcolor}%
  \hl{\textbf{TODO:} #1}%
  \endgroup
}

\newcommand{\discuss}[1]{\textcolor{orange}{\textbf{#1}}}
\newcommand{\upd}[1]{\colorbox{yellow}{#1}}
\newcommand{\expect}{\mathop{\mathbb{E}}\nolimits}
%%% Editing comments.
% ignore this
\newcommand{\ignore}[1]{}
\newcommand{\none}[1]{}
\newcommand{\com}[1]{}
% comment
\newcommand{\cmt}[1]{\begin{sloppypar}\large\textcolor{red}{#1}\end{sloppypar}}
%%% Frequently used terms.
\ifthenelse{\equal{\conf}{cvpr}}
{
}
{
\newcommand{\etal}{{\it{et~al.}}}
\newcommand{\ie}{i.e.,}
\newcommand{\eg}{e.g.,}
}

\ifthenelse{\equal{\conf}{iclr}}
{
}
{
\DeclareMathOperator*{\argmin}{arg\,min}
\DeclareMathOperator*{\argmax}{arg\,max}
}
\DeclareMathOperator*{\minimize}{minimize}
